\documentclass[a4paper,12pt]{article}
\usepackage[usenames,dvipsnames]{color}
\usepackage[hidelinks]{hyperref}
\usepackage{fancyhdr}
\usepackage[flushmargin]{footmisc}
\usepackage{mathptmx} % Added for Times New Roman-like font
\usepackage[version=4]{mhchem}
\usepackage[
  a4paper,
  top = 1.70cm,
  bottom = 1.70cm,
  left = 1.70cm,
  right = 1.70cm
]{geometry}
\renewcommand{\baselinestretch}{1.3}
\renewcommand{\headrulewidth}{0pt}
\renewcommand{\footrulewidth}{0pt}
\setlength{\footskip}{30pt} % Fixes \footskip warning
\setlength{\skip\footins}{20pt} % fixed gap, no stretch/shrink
\pagestyle{fancy}
\fancyhf{} % clear all header and footer fields
\fancyfoot[C]{\thepage} % Center page number in the footer
\hypersetup{
    colorlinks=true,
    linkcolor=RoyalBlue,
    urlcolor=RoyalBlue,
    citecolor=RoyalBlue,
    anchorcolor=RoyalBlue
}
\urlstyle{same}
\input{glyphtounicode}
%-------------------------------------------
% STARTS HERE
\begin{document}
%HEADING
\begin{center}
    {\Huge \scshape {\fontsize{25}{30}\selectfont{Reza}} {\fontsize{25}{30}\selectfont{\textbf{Aliasgari Renani}}}} \\ \vspace{3pt}
    {\small \raisebox{-0.2\height}\ {10/10/2025}} $|$
    {\small \raisebox{-0.2\height}\ {Motivation Letter}} $|$
    {\small \raisebox{-0.2\height}\ {Physics and Astronomy}} $|$
    {\small \raisebox{-0.2\height}\ {Aarhus University}} $|$
    {\small \raisebox{-0.2\height}\ {PhD Position}}
    \vspace{-10pt}
\end{center}
\vspace{-10pt}
\noindent\rule{\textwidth}{0.5pt}

My research objective is to contribute to the development of quantum devices in which controlling phases and collective states in solid materials enables circuit-like functionality. To move toward that vision, I want to work on the experimental side of how quantum phases emerge and how they can be switched, steered, or stabilized in real materials. I am currently completing the final year of my Master’s degree in Physics at the Moscow Institute of Physics and Technology (MIPT), where my work on semiconductor devices has given me experience with experimental setups, electronic materials, and solid-state physics. I now want to focus on ultrafast control in quantum materials to understand how these phases can be used in ways that could eventually enable technologies such as quantum logic.

\vspace{10pt}
I joined Dr.\ Koveshnikov's laboratory in the Russian Academy of Sciences over two years ago to study radiation effects on microelectronic structures, focusing on electrical characterization of semiconductors and \ce{SiO2}-based MOS devices. We studied the impact of electron and gamma irradiation and hydrogen plasma treatment on MOS devices, identified traps by location (oxide, semiconductor, interface) and carrier type (electrons, holes). This work has resulted in a first-author publication\footnote{R. Aliasgari Renani, O.A. Soltanovich, M.A. Knyazev, S.V. Koveshnikov, Investigation of low energy electron irradiated \ce{SiO2} based MOS devices by C-V and TSC techniques, Russian Microelectronics, 2023} and provides insight into device stability under bias stress, radiation immunity, and oxide trap density at the dielectric-semiconductor interface. I also developed optimized MATLAB applications to automate the experimental characterization techniques, which reduced run-times and created measurement pipelines. 

\vspace{10pt}
After my undergraduate degree, I joined the System-on-Chip Development Laboratory and the Laboratory of Plasma Systems under the supervision of Dr.\ Vasiliev to develop FPGA-based systems and to study radiation and plasma effects on these structures. I implemented video-processing algorithms by manual Verilog coding and by generating HDL from Simulink models. I verified functionality of the algorithms using testbenches and Python automation scripts. In the Plasma Lab, we ran experiments where an FPGA with an attached camera and a synthesized image-processing design was placed in an electron-beam plasma system with low-pressure oxygen and exposed to electron irradiation and X-rays generated from a tungsten target. The FPGA output signal was monitored while thermal cycling was applied and preliminary functionality tests were performed. 

\vspace{10pt}
My research on the effects of irradiation on semiconductor devices and the resulting insight into defect formation and lattice-level disorder has motivated me to pursue further study in the Department of Physics and Astronomy at Aarhus University. The PhD position on controlling symmetry breaking in quantum materials with ultrafast correlated disorder is an ideal opportunity to extend my experience and investigate how external perturbations can influence electronic phases in quantum materials. I would be honored to have the opportunity to join Dr. Wall’s group and learn how femtosecond pulses across the electromagnetic spectrum can control quantum materials on their natural timescales. My background in electrical characterization and harsh-condition experiments has prepared me to contribute to this research.

\end{document}
% ENDS HERE
%-------------------------------------------
