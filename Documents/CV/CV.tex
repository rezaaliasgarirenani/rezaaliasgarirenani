%-------------------------
% Resume in Latex
% Author: Reza Aliasgari Renani (reordered to reverse-chronological)
% License: MIT
%------------------------

\documentclass[a4paper, 11pt]{article}
\usepackage{titlesec}
\usepackage{marvosym}
\usepackage[usenames,dvipsnames]{color}
\usepackage{enumitem}
\usepackage[hidelinks]{hyperref}
\usepackage{fancyhdr}
\usepackage[english]{babel}
\usepackage{tabularx}
\usepackage{fontawesome5}
\usepackage{multicol}
\usepackage[version=4]{mhchem}
\usepackage[
  a4paper,
  top    = 1.70cm,
  bottom = 1.70cm,
  left   = 1.70cm,
  right  = 1.70cm
]{geometry}
\usepackage{mathptmx} % Added for Times New Roman-like font

\renewcommand{\baselinestretch}{1.1}
\renewcommand{\headrulewidth}{0pt}
\renewcommand{\footrulewidth}{0pt}

% Defining custom commands
\newcommand{\resumeItem}[1]{
    \item\small{
        {#1 \vspace{-2pt}}
    }
}

\newcommand{\resumeSubheading}[4]{
    \vspace{-2pt}\item
    \begin{tabular*}{1.0\textwidth}[t]{l@{\extracolsep{\fill}}r}
        \textbf{#1} & \textbf{\small #2} \\
        \small#3 & \small #4
    \end{tabular*}\vspace{-7pt}
}

\newcommand{\resumeProjectHeading}[2]{
    \item
    \begin{tabular*}{1.001\textwidth}{l@{\extracolsep{\fill}}r}
        \small#1 & \textbf{\small #2}\\
    \end{tabular*}\vspace{-7pt}
}

\renewcommand\labelitemi{$\vcenter{\hbox{\tiny$\bullet$}}$}
\renewcommand\labelitemii{$\vcenter{\hbox{\tiny$\bullet$}}$}

\newcommand{\resumeSubHeadingListStart}{\begin{itemize}[leftmargin=0.0in, label={}]}
\newcommand{\resumeSubHeadingListEnd}{\end{itemize}}
\newcommand{\resumeItemListStart}{\begin{itemize}}
\newcommand{\resumeItemListEnd}{\end{itemize}\vspace{-5pt}}

\setlength{\multicolsep}{-3.0pt}
\setlength{\columnsep}{-1pt}

\hypersetup{
    colorlinks=true,
    linkcolor=RoyalBlue,
    urlcolor=RoyalBlue,
    citecolor=RoyalBlue,
    anchorcolor=RoyalBlue
}

\input{glyphtounicode}

\pagestyle{fancy}
\fancyhf{} % clear all header and footer fields
\fancyfoot{}
\setlength{\footskip}{30pt} % Fixes \footskip warning
\urlstyle{same}
\raggedbottom
\raggedright
\setlength{\tabcolsep}{0in}

% Formatting section titles
\titleformat{\section}{
    \vspace{-4pt}\scshape\raggedright\large\bfseries
}{}{0em}{}[\color{black}\titlerule \vspace{-5pt}]

\pdfgentounicode=1

% Starting the document
\begin{document}


%----------HEADING----------
\begin{center}
    {\Huge \scshape {\fontsize{25}{30}\selectfont{Reza}} {\fontsize{25}{30}\selectfont{\textbf{Aliasgari Renani}}}} \\ \vspace{3pt}
    \small ~ 
    \href{mailto:rezaaliasgarirenani@gmail.com}{\raisebox{-0.2\height}\faEnvelope\ \underline{rezaaliasgarirenani@gmail.com}} $|$ ~
    \href{https://orcid.org/0009-0000-8983-755X}{\raisebox{-0.2\height}\faOrcid\ \underline{ORCID}} $|$ ~
    \href{https://www.researchgate.net/profile/Reza-Aliasgari-Renani}{\raisebox{-0.2\height}\faResearchgate\
    \underline{ResearchGate}} $|$ ~
    \href{https://scholar.google.com/citations?user=L9Vv3C8AAAAJ&hl=en}{\raisebox{-0.2\height}\faGraduationCap\ \underline{Google Scholar}} $|$ ~
    \href{https://github.com/rezaaliasgarirenani}{\raisebox{-0.2\height}\faGithub\ \underline{GitHub}} $|$ ~
    \href{https://linkedin.com/in/reza-aliasgari-renani}{\raisebox{-0.2\height}\faLinkedin\ \underline{LinkedIn}}
    \vspace{-10pt}
\end{center}

%-----------EDUCATION-----------
\section{\textcolor{red}{Ed}ucation}
\resumeSubHeadingListStart
    %--- Most recent (M.Sc.) first ---
    \resumeSubheading
        {\href{https://mipt.ru/}{Moscow Institute of Physics and Technology (MIPT, Phystech)}}{September 2024 -- June 2026}
        {\textbf{M.Sc.} in Applied Mathematics and Physics, Program: Plasma Physics, GPA: 4.7/5.0}{Moscow, Russian Federation}
        {\resumeItem{\textbf{Thesis title:} Investigation of radiation induced effects on FPGA-based signal processing systems for space applications.}}
    %--- Earlier (B.Sc.) after ---
    \resumeSubheading
        {\href{https://mipt.ru/}{Moscow Institute of Physics and Technology (MIPT, Phystech)}}{September 2020 -- June 2024}
        {\textbf{B.Sc.} in Technical Physics, Program: Aerospace Technology, GPA: 4.56/5.0}{Moscow, Russian Federation}
        {\resumeItem{\textbf{Thesis title:} Investigation of the effects of low energy (1 - 20 keV) electrons and high energy (1 MeV) gamma quanta irradiation on the electro-physical properties of dielectric-semiconductor structures.}}
\resumeSubHeadingListEnd
\vspace{-15pt}

%-----------EXPERIENCE-----------
\section{{\textcolor{red}{Re}search} Experience}
\resumeSubHeadingListStart
    %--- Current / Most recent position first ---
    \resumeSubheading
        {\href{http://ai.mipt.ru/design-center}{\parbox[t]{\dimexpr\linewidth-6cm}{Design Center for the Development of Microprocessor Technology for AI Systems, System-on-Chip Development Laboratory}}}
        {September 2024 -- Present}
        {\textbf{Programmer / RTL Design Engineer}}{Moscow, Russian Federation}
        \resumeItemListStart
            \resumeItem{\textbf{DSP Implementation} \\ Ported mathematical algorithms into efficient Verilog implementations. Built a fixed-point library and LUT-based function approximations (Horner’s method) to support fixed-point computations. Implemented image processing algorithms (rgb2hsv, color segmentation, Sobel edge detection, global tone-mapping, frame summing, demosaicing, bird’s-eye view, fisheye correction) via Simulink HDL code generation and manually written Verilog. \\ Optimized latency and throughput, resolved synchronization and pipelining issues.}
            \resumeItem{\textbf{Simulation, Verification and Synthesis} \\ Created comprehensive testbenches in Verilog and used Python for simulation automation and data analysis to verify DSP functionality and performance. Synthesized, mapped, and routed HDL code using Vivado (FPGA: Xilinx Artix-7, Zybo Z7). Verified ISP algorithms by first streaming test images via HDMI from a host computer, and subsequently with a live camera connected to the FPGA.}
           \resumeItem{\textbf{Investigation of FPGA devices under electron-beam plasma exposure} \\ Conducted irradiation experiments on FPGA boards using electron beams (25 -- 60\,keV, up to 100\,mA) in low-pressure oxygen atmospheres ($10^{-6}$ -- 50\,Torr), generating plasma and X-rays. Applied combined thermal cycling ($218$--$393$\,K) and surface charging to evaluate FPGA reliability under radiation- and plasma-induced stress.}

        \resumeItemListEnd

    %--- Previous (concurrent) roles ---
    \resumeSubheading
        {\href{https://new.ras.ru/en/}{Institute of Microelectronics Technology, Russian Academy of Sciences}}{March 2023 -- August 2024}
        {\textbf{Laboratory Researcher}}{Moscow, Russian Federation}
        \resumeItemListStart
            \resumeItem{\textbf{Experimental Equipment Installation and Automation} \\ Installed experimental devices including Everbeing Cryo-station (80K – 450K) with 4 micromanipulators, Lakeshore Temperature Controller Model 336, Keithley SourceMeter 2450, Parametric Analyzer Keithley 4200A-SCS, Keysight Electrometer B2987A, Aktakom 3048, and Zurich Instruments MFIA Impedance Analyzer. Developed {\href{https://github.com/rezaaliasgarirenani/IMT-Automation}{applications}} in MATLAB to automate experimental techniques: Thermally Stimulated Current, Capacitance-Voltage, Current-Voltage, Current-Time and Deep-Level Transient Spectroscopy.}
            \resumeItem{\textbf{Theoretical Investigation} \\ Developed theoretical understanding and studied experimental techniques for semiconductor devices (MOS, MOSFET, diode, RRAM). Predicted sample behavior, interpreted physical phenomena, and determined measurement parameters.}
            \resumeItem{\textbf{Experimental Investigation and Data Processing} \\ Conducted electrical characterization experiments on microelectronic structures, processed data using MATLAB and Origin Pro, removed extraneous random telegraph noise points, and compared results with theoretical models.}
        \resumeItemListEnd
    \resumeSubheading
        {\href{https://mipt.ru/dasr/about/kaf_faculty/mmsp}{Laboratory of Modeling of Mechanical Systems and Processes}}{March 2023 -- August 2024}
        {\textbf{Engineer / Technician}}{Moscow, Russian Federation}
        \resumeItemListStart
            \resumeItem{\textbf{Engineering Design and Development} \\ Designed, developed, and analyzed models for a CubeSat orbital deployer and vibration fixture using SolidWorks. Created multiple prototypes, which passed random vibration simulation and dynamic analysis, and tested them on the UVE 4000 vibro-stand for mechanical environmental factors and vibration resistance.}
        \resumeItemListEnd
        
\resumeSubHeadingListEnd
\vspace{-15pt}

%-----------PUBLICATIONS---------------
\section{{\textcolor{red}{Pu}blications} \& Conferences}
\resumeSubHeadingListStart
    \item \underline{R. Aliasgari Renani}, O.A. Soltanovich, M.A. Knyazev, S.V. Koveshnikov. \\ \textit{Investigation of low energy electron irradiated \ce{SiO2} based MOS devices by capacitance-voltage and thermally stimulated current techniques.} \href{https://doi.org/10.1134/S1063739723600516}{Journal Paper}, Russian Microelectronics, 2023
    \item \underline{R. Aliasgari Renani}, O.A. Soltanovich, M.A. Knyazev, S.V. Koveshnikov. \\ \textit{Study of \ce{SiO2} based MOS by capacitance-voltage and thermally stimulated current techniques.} \href{https://icmne.ftian.ru/wp-content/uploads/icmne-2023_e-version.pdf}{Presentation}, p.122. The 15th International Conference Micro- And Nanoelectronics (\href{https://icmne.ftian.ru}{ICMNE 2023}).
    \item \underline{R. Aliasgari Renani}, O.A. Soltanovich, M.A. Knyazev, S.V. Koveshnikov. \\ \textit{Investigation of electrically active defects introduced into silicon oxide by irradiation of low-energy electrons, by methods of Capacitance-Voltage characteristics and thermally-stimulated current.} \href{https://cebt23.iptm.ru/download/numbered/91.pdf}{Poster}, Second Joint Conference on Electron Beam Technologies and X-ray Optics in Microelectronics (\href{https://cebt23.iptm.ru}{CALT 2023})
    \item \underline{R. Aliasgari Renani}, V. Vasilevskiy, V. Vologin, V. Chesnokov \\ \textit{Comparative analysis of manual Verilog and Simulink-generated HDL code for image processing algorithms.} Forthcoming presentation, Yadro FPGA Systems Conference, 2025.
    
\resumeSubHeadingListEnd
\vspace{-20pt}

%-----------TECHNICAL SKILLS-----------
\section{{\textcolor{red}{Te}chnical Skills}}
\begin{center}
    \begin{minipage}[t]{0.45\textwidth}
        \textbf{Advanced}
        \vspace{-7pt}
        \begin{itemize}
            \setlength{\itemindent}{0em}
            \setlength{\itemsep}{-3pt}
            \item Automation of Experimental Techniques
            \item Electrical Characterization
            \item Data Processing
            \item MATLAB, Simulink, HDL Coder
            \item Verilog, RTL Design
            \item FPGA Development
            \item Fixed-Point Computations
            \item SciPy, NumPy, OpenCV
        \end{itemize}
    \end{minipage}%
    \hfill
    \begin{minipage}[t]{0.45\textwidth}
        \textbf{Intermediate}
        \vspace{-7pt}
        \begin{itemize}
            \setlength{\itemindent}{0em}
            \setlength{\itemsep}{-3pt}
            \item Python, C++, Arduino
            \item Vivado, Vitis
            \item Git, Unix/Linux OS
            \item OriginLab
            \item SolidWorks
            \item ERDAS IMAGINE
            \item PCB, EasyEDA
            \item OpenRocket
        \end{itemize}
    \end{minipage}
\end{center}
\vspace{-15pt}

%-----------PROJECTS-----------
\section{{\textcolor{red}{Pr}ojects}}
    %--- Most recent projects first ---
    {\href{https://github.com/icarus-imperium/rocket-2025}{\textbf{Model Rocket}}} \hfill \textbf{April 2023 and 2025} \\
    \begin{itemize}
        \vspace{-9pt}
    \item Constructed model rockets with 40 and 60 Newton-second impulses as part of a team.
        \vspace{-9pt}
    \item  Launched three model rockets over two years at the Cosmonautics Day of MIPT.
    \end{itemize}
    \vspace{-9pt}
    {\href{https://github.com/rezaaliasgarirenani/Rover}{\textbf{Model Lunar Rover}}} \hfill \textbf{June 2023}
    \begin{itemize}
        \vspace{-9pt}
    \item  Collaborated on a machine capable of navigating obstacles without round wheels.
        \vspace{-9pt}
    \item  Tested multiple prototypes, with the final design approved by the laboratory head.
    \end{itemize}
    \vspace{-9pt}
    {\href{https://github.com/rezaaliasgarirenani/Aircraft-Detection-System}{\textbf{Aircraft Detection System}}} \hfill \textbf{June 2022}
    \begin{itemize}
        \vspace{-9pt}
    \item  Investigated and applied algorithms to detect aircraft using photoresistors and transistors.
        \vspace{-9pt}
    \item  Developed a system capable of rotational and translational movement to track aircraft.
    \end{itemize}
    \vspace{-9pt}
    {\href{https://github.com/rezaaliasgarirenani/Non-Conservative-Electric-Fields-and-Voltmeters}{\textbf{Investigation of Non-Conservative Electric Fields and Voltmeters}}} \hfill \textbf{May 2022} \\
    \begin{itemize}
        \vspace{-9pt}
    \item  Designed an experimental setup to analyze position-dependency of voltmeter readings in parallel circuits.
        \vspace{-9pt}
    \item  Demonstrated the non-intuitive potential differences generated by changing magnetic fields.
    \end{itemize}
    \vspace{-15pt}

%-----------AWARDS-----------
\section{{\textcolor{red}{Aw}ards}}
\begin{itemize}
    %--- Most recent award first ---
     \resumeItem{\textbf{Recipient:} Full State Russian Scholarship for Foreign Students, MIPT} \hfill \textbf{September 2024}
    \resumeItem{\textbf{Recipient:} Iranian State Scholarship, Isfahan University of Technology} \hfill \textbf{September 2019}
    \resumeItem{\textbf{Awardee:} Participant of the \href{https://congress.aero/en/}{5th and 7th Eurasian Aerospace Congress}} \hfill \textbf{July 2023 and 2025}

\end{itemize}
\vspace{-15pt}

%-----------LANGUAGES-----------
\section{{\textcolor{red}{La}nguages}}
\begin{center}
    \textbf{English}: C2 (TOEFL iBT 113) \quad $|$ \quad 
    \textbf{Russian}: B1 \quad $|$ \quad 
    \textbf{German}: B1 (ÖSD) \quad $|$ \quad 
    \textbf{Farsi}: Native
\end{center}

\end{document}
